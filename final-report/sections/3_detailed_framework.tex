\documentclass[../main.tex]{subfiles}

\begin{document}

\subsection{High-Level Overview}

\subsection{Languages}

I use C for the core dictionary because it runs faster and provides more precise memory control. I initially wrote it in Python, but it took 3 seconds to launch and violated the time constraint. The bottleneck turns out to be CPU computation as opposed to disk IO. Moving to C should effectively speed it up since compiled languages typically compute much faster than interpreted languages.

I use Python for the story writer because it is easier to code, supports regular expression and features various sampling methods. Usage of these functionality is described in <Section>. Python libraries such as NumPy have a mature and efficient C/Fortran back-end. Compared to reinvented wheels, they are faster, more robust and easier to debug. Moreover, exception mechanism in Python makes it simpler to handle special cases that appear in a natural language.

\subsection{Data Structures}

\subsection{Software Implementation}

\end{document}