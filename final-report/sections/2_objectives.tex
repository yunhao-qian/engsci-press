\documentclass[../main.tex]{subfiles}

\begin{document}

The core dictionary program should:

\begin{enumerate}
	\item \textit{Launch and response fast.}
	A slow start-up puts the user in a bad mood even before s/he starts using the program.
	
	\textbf{Metric:} Measure the time interval between a user request and its response. Shorter time in seconds is better. Start-up should take less than 1 second.
	
	\item \textit{Use memory efficiently.}
	Users might run the program on an outdated computer or a virtual machine, which typically has very limited memory. Large memory use hurts performance and can cause system failure.
	
	\textbf{Metric:} Measure the increased memory usage after loading the same dictionary dataset. Less memory in megabytes is better.
	
	\item \textit{Add dictionary entries easily.}
	The provided data have a lot of typos. Users like me might be unsatisfied and want to customize them. After following a clear and simple procedure, users should be able to add data files with the same format.
	
	\textbf{Metric:} Count the number of operations to load a CSV file into the dictionary dataset. Fewer operations are better.
\end{enumerate}

The story writer program should:

\begin{enumerate}
	\item \textit{Produce grammatically correct sentences.}
	To generate meaningful and logical stories is beyond my ability. To tell my story writer apart from a monkey hitting keys, the only way is to force my production grammatically correct.
	
	\textbf{Metric:} Copy and paste the produced text into Microsoft Word. Green underlines flag grammatical errors. Fewer grammatical errors per sentence are better.
	
	\item \textit{Control the length of generated text accurately.}
	Sentence generation is slow, so it is a waste of time to work on unneeded sentences.
	
	\textbf{Metric:} Calculate the percentage difference between the user-specified length and the length of generated text. Smaller average difference is better.
\end{enumerate}

\end{document}