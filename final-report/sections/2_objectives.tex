\documentclass[../main.tex]{subfiles}

\begin{document}

The core dictionary program should:

\begin{enumerate}
	\item \textit{Launch and response fast.}
	A slow start-up reduces user experience.
	
	\textbf{Metric:} Measure the time interval between a user request and its response. A shorter interval would be ideal. The start-up should take less than 1 second \cite{bib:1s_time_limit}.
	
	\item \textit{Use memory efficiently.}
	Users might run the program on an outdated computer or a virtual machine, which usually has very limited memory. The excessive use of memory will impact the performance negatively and cause a system failure.
	
	\textbf{Metric:} Measure the increased memory usage after loading the same dictionary dataset. Less memory in megabytes is better.
	
	\item \textit{Add dictionary entries easily.}
	The provided data have a lot of typos. Users might be unsatisfied, thus they want to customize them. After following a clear and simple procedure, users should be able to add data files with the same format.
	
	\textbf{Metric:} Count the number of operations to load a \texttt{.csv} file into the dictionary dataset. Fewer operations are better.
\end{enumerate}

The story writer program should:

\begin{enumerate}
	\item \textit{Produce grammatically correct sentences.}
	To generate meaningful and logical stories is beyond my ability. To tell my story writer apart from a monkey hitting keys, the only way is to make my production grammatically correct.
	
	\textbf{Metric:} Copy and paste the produced text into Microsoft Word. Green underlines flag grammatical errors. Fewer grammatical errors per sentence are better.
	
	\item \textit{Control the length of the generated text accurately.}
	The process of sentence generation is slow. It will be a waste of time to work on unneeded sentences.
	
	\textbf{Metric:} Calculate the percentage difference between the length specified by the user and the length of the generated text. Smaller average difference is better.
\end{enumerate}

\end{document}